% !TeX spellcheck = en_GB
\documentclass[12pt,fleqn]{article}

\usepackage[english]{babel}
\usepackage{SpeedyGonzales}
\usepackage{MediocreMike}
%\usepackage{Blastoise}

\title{Image Segmentation}
\author{}
\date{\today}

\pagestyle{fancy}
\fancyhf{}
\lhead{}
\chead{}
\rhead{}
\rfoot{Side \thepage{} af \pageref{LastPage}}

\graphicspath{{Billeder/}}
\linespread{1.15}


%\numberwithin{equation}{section}
%\numberwithin{footnote}{section}
%\numberwithin{figure}{section}
%\numberwithin{table}{section}

\begin{document}

%\maketitle
%\thispagestyle{fancy}

\begin{titlepage}
	\begin{center}
		\textsc{\LARGE Image Segmentation}\\
		[1.0cm]
		{
		\large
		\begin{tabular}{lr}
			Anders Henriksen&s183904\\
			Asger Schultz&s183912\\
			Oskar Wiese&s183917\\
			Mads Andersen&\\
			Søren Winkel Holm&s183911
		\end{tabular}
		}\\
		[0.5cm]
		\textsc{\large \today}
	\end{center}
\end{titlepage}
\tableofcontents \newpage


\section{Introduction}
One of the key motivation in choosing to work with deep neural networks, more specifically
SegNet, is the adaptability of the algorithm as well as the wide application. SegNet can be
used to segment images to speed up manual work, such as cell assessment by a doctor for
cancer cells or counting invasive bird species. Using SegNet in autonomous vehicles is also
highly desirable in order to classify objects such as people, houses, trees etc. In this project SegNet is used to classify crops in arieal drone images. 

\section{Methods}
Netværket blev initialiseret

\section{Results}

\section{Discussion}

\end{document}



















