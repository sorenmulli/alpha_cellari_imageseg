% !TeX spellcheck = en_GB
% Template for ICASSP-2010 paper; to be used with:
%          mlspconf.sty  - ICASSP/ICIP LaTeX style file adapted for MLSP, and
%          IEEEbib.bst - IEEE bibliography style file.
% --------------------------------------------------------------------------
\documentclass{article}
\usepackage{amsmath,graphicx,02460}
\usepackage{url}

\toappear{02460 Advanced Machine Learning, DTU Compute, Autumn 2019}
\newcommand{\pro}{\ensuremath{\ \%}}

% Example definitions.
% --------------------
\def\x{{\mathbf x}}
\def\L{{\cal L}}

% Title.
% ------
\title{SegNet on Drone Images: Image Segmentation for Smart Agriculture}
%
% Single address.
% ---------------
\name{			Anders Henriksen, Asger Schultz, Oskar Wiese, 	Mads Andersen,
	Søren Winkel Holm }
\address{s183904, s183912, s183917, s173934, s183911}
%
% For example:
% ------------
%\address{School\\
%	Department\\
%	Address}
%
% Two addresses (uncomment and modify for two-address case).
% ----------------------------------------------------------
%\twoauthors
%  {A. Author-one, B. Author-two\sthanks{Thanks to XYZ agency for funding.}}
%	{School A-B\\
%	Department A-B\\
%	Address A-B}
%  {C. Author-three, D. Author-four\sthanks{The fourth author performed the work
%	while at ...}}
%	{School C-D\\
%	Department C-D\\
%	Address C-D}
%
\begin{document}
%\ninept
%

\maketitle
%
\begin{abstract}
The abstract should appear at the top of the left-hand column of text, about
0.5 inch (12 mm) below the title area and no more than 3.125 inches (80 mm) in
length.  Leave a 0.5 inch (12 mm) space between the end of the abstract and the
beginning of the main text.  The abstract should contain about 100 to 150
words, and should be identical to the abstract text submitted electronically
along with the paper cover sheet.  All manuscripts must be in English, printed
in black ink.
\end{abstract}
%
\begin{keywords}
One, two, three, four, five
\end{keywords}
%
\section{Introduction}
\label{sec:intro}

\subsection{Dataset and preprocessing}
Our dataset consists of two large, high resolution orthomosaic RGB images of a sugar cane field\footnote{Aerial image: http://www.lapix.ufsc.br/wp-content/uploads/2019/05\\/sugarcane2.png\\
Ground truth: http://www.lapix.ufsc.br/wp-content/uploads/2019/05\\/crop6GT.png}.
The first image consists of several drone images stitched together, with the second image of the same size being the by an expert biologist manually labelled human ground truth.
The three classes each have a corresponding colour -- crop rows are green, weeds are yellow, and soil is red.
Void pixels are black.

\subsection{Motivation}
Image segmentation and object classification have been a huge talking point in recent years. This is partly due to the wide array of possible applications and the recent interest in machine learning and, in particular, deep learning. One such important application is seperating weeds, crops and dirt in aerial drone images of a field. Applying the SegNet architecture to this field of crop segmentation could allow for smart agriculture that circumvents the many laborious man-hours of manual classification. Using SegNet for this classification could also prove useful in terms of the computational efficiency, as this allows for end-to-end training and makes embedded systems, e.g. in drones, feasible.

\subsection{Goal and application}


In the following segments of this paper, the relevant methods, results and discussion will be covered; In section 2, methods such as regularization, data augmentation, metrics and chosen loss function are described. The results are tabulated in section 3. Section 4 is a discussion of the results and accuracy of the reconstruction, a comparison to other similar methods as well as a perspective on the problem. The relevant references are included in section 5.

\section{METHODS}
\label{sec:format}

\subsection{Preprocessing and data augmentation}
In order to get the most out of the data, preprocessing is needed.
First, the RGB values of the non-void pixels of the aerial image are standardized, and a matrix is used to represent the ground truth.
Each entry corresponds to a pixel and contains a number 0-2 for the different classes or 3 for void.
The images are then padded with black pixels and cropped into smaller $ 512\times 512 $ pixel images.
Any of these images containing only black pixels are discarded, leaving a total of 108 pairs of aerial photo/ground truth images.
These are then split into 69 training, 18 validation, and 25 test images.

In order to increase the effective size of the dataset, we perform aggressive data augmentation.
When training the network, each pair of aerial photo/ground truth images is randomly cropped into $ 256\times 256 $ pixel images.
Furthermore, we applied a 50\pro\ chance of performing a top/down flip as well as a 50\pro\ chance of left/right flip to each image pair.
Even though data augmentation is not as a good as more independent data, it still allows the network to generalize better and overfit less.

\subsection{Regularization}
Because deep neural networks are so flexible models, regularization is necessary on top of the data augmentation to further reduce overfitting.
This is done in two ways.

Dropout at 10\pro\ is used after each blue block in the network (see Fig. \ref{fig:arch}).
This randomly shuts off nodes during training leading to node redundancy and variability, as the same input will vary somewhat in its output, which learns the network to generalize better.
We experimented with higher dropout, but found that too much would significantly reduce learning.

Batch normalization is applied after each dropout.
This standardizes the activations, keeping them close to zero.
As a result, the weights and biases also stay close to zero, which reduces the flexibility of the model, leading to less overfitting.
Batch normalization also has several other benefits, such as reducing the vanishing gradient problem and allowing for a higher learning rate and thus faster convergence time. \cite{bn}
\\
\\
\subsection{}
The encoder part of the network creates a rich feature map representing the image content. The more 
layers of max-pooling there are the more translation invariance for robust 
classification can be achieved. The boundary detail is very important when 
dealing with image segmentation. Hence, capturing boundary information in 
the feature maps of the encoder before upsampling is important. This can 
simply be done by storing the whole feature map, but due to memory 
constrains only the maxpooling indices are saved, which is a good 
approximation of the feature maps. 
\subsection{Loss function: Quality over quantity}
Multi-class cross entropy because:
\begin{itemize}
	\item Softmax Network: Minus log likelihood
	\item Can be seen as a classic multiclass classifier -- just on a pixel-by-pixel basis.
\end{itemize}
Weighted cross entropy because:
\begin{itemize}
	\item Unbalanced class distribution: Network has to learn to focus on important pixels: Don't classify everything as dirt.
	\item Initial tests made the network behave as the baseline: Simple features in early layers got were not penalized enough and learning was not stable.
	\item Resampling expensive
\end{itemize}
\subsection{Metrics}
Had to use different metrics because
\begin{itemize}
	\item Not agreement in Image Segmentation papers.
	\item Want to get accuracy on a global scale and on a class scale.
	\item Different metrics important in different fields.
\end{itemize}
The metrics \footnote{https://hal.inria.fr/hal-01581525/document}\footnote{
	http://www.bmva.org/bmvc/2013/Papers/paper0032/paper0032.pdf} 
\begin{itemize}
	\item Global accuracy: Trivial and  not very important because of class imbalance but is good for smoothness
	\item Mean class-wise accuracy: Takes class imbalance into account. Is what is being optimized for in the model.
	\item Mean Intersect over Union: "Jaccard Index".  Found to be better correlated with human classification though still only \(\approx 0.5\). Favours region smoothness highly and not boundary accuracy.
	\item Harmonic mean of precision and recall. To compare to others with same project. Penalizes false positives and gives less credit to true negatives thus being better for unbalanced classes.
\end{itemize}

\subsection{Regularization and Hyperparameters}
Regularization
\begin{itemize}
	\item NN's are prone to overfitting, because they are so flexible
	\item Prevent overfitting $ \to $ better results on test data
	\item Three methods
	\item Dropout: Randomly remove nodes to increase variability. $ p=10 \%$
	\item Data augmentation: Increase size of dataset
	\begin{itemize}
		\item Crop each $ 512\times 512 $ to random $ 256\times 256 $
		\item 50\% chance of flip T/D and 50\% chance of flip L/R
	\end{itemize}
	\item Batch normalization normalizes activations
	\begin{itemize}
		\item Faster convergence
		\item Prevents ReLU from not learning
		\item Introduces noise
		\item Reduces vanishing/exploding gradient problem, as values stay close to 0
	\end{itemize}
\end{itemize}
Hyperparameters
\begin{itemize}
	\item Adaptive learning rate from ADAM optimizer, initialized at $ 2\cdot10^{-4} $
	\item Total: 26 conv + batchnorm + ReLU with dropout, 5 pool/upsample, 1 softmax
	\item 14.7 M parameters in encoder -- significantly lower than 134 M in VGG16 because of no fully-connected layers
	\item Kernel size: $ 3\times 3 $, stride 1, maxpool: $ 2\times 2 $, stride 2
	\item Corresponding padding of 1 to prevent reduction of image size
\end{itemize}



\subsection{Unification of cropped image predictions}
In a real-world application of the field classification a farmer would want a complete and precise segmentation of his whole field at once, such that fertilizer and pesticides can be distributed accordingly. To accomplish this, a reconstruction of the smaller image inferences is neccesary. The most straight forward method of combining the smaller images, by simply lining them up next to each other results in a very rough transition between the smaller inferences. This can be seen in the left side of \ref{fig:earlylatereconstruction}. The blocky nature of the field prediction is caused by a lack of information from neighbouring pixel when inference is performed near the borders of an image. To solve this problem, we have chosen to increase the size of the cropped images and add some overlap, and then infer on these enlarged pictured. In the procedure of joining the enlarged cropped pictures they are cropped again, to avoid the border areas. At the cost of computational efficiency more information is available near the borders and a smooth field prediction can be archieved. This can be seen in the right side of \ref{fig:earlylatereconstruction}. A visualization of the reconstruction technique can be seen in appendix \ref{reconstruction_technique}.


\begin{figure}
	\centering
	\includegraphics[width=0.9\linewidth]{early_late_reconstruction2}
	\caption{\textbf{Left:} Smaller inferences put next to eachother without use of reconstruction techniques. \textbf{Right:} Reconstruction with padding and overlap between smaller inferences.}
	\label{fig:earlylatereconstruction}
\end{figure}





\section{RESULTS}
\label{sec:illust}

% Below is an example of how to insert images. Delete the ``\vspace'' line,
% uncomment the preceding line ``\centerline...'' and replace ``imageX.ps''
% with a suitable PostScript file name.
% -------------------------------------------------------------------------
%\begin{figure}[htb]

%\begin{minipage}[b]{1.0\linewidth}
  %\centering
%  \centerline{\includegraphics[width=8.5cm]{image1}}
%  \vspace{2.0cm}
 % \centerline{(a) Result 1}\medskip
%\end{minipage}
%
%\begin{minipage}[b]{.48\linewidth}
  %\centering
%  \centerline{\includegraphics[width=4.0cm]{image3}}
%  \vspace{1.5cm}
 % \centerline{(b) Results 3}\medskip
%\end{minipage}
%\hfill
%\begin{minipage}[b]{0.48\linewidth}
  %\centering
 % \centerline{\includegraphics[width=4.0cm]{image4}}
%  \vspace{1.5cm}
 % \centerline{(c) Result 4}\medskip
%\end{minipage}
%
%\caption{Example of placing a figure with experimental results.}
%\label{fig:res}
%
%\end{figure}

% To start a new column (but not a new page) and help balance the last-page
% column length use \vfill\pagebreak.
% -------------------------------------------------------------------------
%\vfill
%\pagebreak


\section{DISCUSSION}
\label{sec:foot}
\subsection{Comparison of different image segmentation neural networks}
\begin{itemize}
	\item Several competing network structures with high performance in image segmentation. U-net, FCN, DeepLabv1, DeconvNet
	\item Purpose of SegNet, efficient
	\item 3 out of the 4 mentioned uses the encoder from the famous VGG16 paper, but differ in decoder.
	\item FCN, No decoder -> Blocky segmentation, but very efficient in inference time.
	\item DeconvNet, Deconvolution and fully connected layers. 
	\item U-Net, (different purpose), skip connections. 
	\item Main takeaway  
	\item (Deeplabv-LargeFOV \& FCN)
	
\subsubsection{Draft for comparison}
Before we move on to a comparison of the leading network architectures in the field of image segmentation, It is important to emphasize that the purpose of SegNet is to be efficient in inference time and memory wise, such that it can be used in embedded systems. For example in drones or in self-driving cars.  Besides SegNet the most acknowledged structures are named U-net, DeepLabv1, DeconvNet and FCN. They have very similar performance and the encoding architecture is almost identical. But they differ in decoding structure and techniques and as a common denominator the SegNet alternatives uses a lot of memory during inference. Among the heavy users of memory are the fully connected layers of the DeconvNet and FCN, and the skip connections in the U-Net. And in comparison, with the other networks, SegNet uses half the memory during inference and comes in second with regard to inference time. The fastest network is the FCN, and this is becaused it does not have a decoder structure, but makes a prediction directly from the output from the encoder, and this leads to blocky predictions.

\subsection{Potential in SegNet}
The architecture of SegNet allows for an easy and fast end to end training of the network. As a consequence of this SegNet is easily modified into other use cases without much technical effort. However, it can be challenging and costful to acquire labelled data. In our project we only have a single field image, which may or may not be representative of the field in another lighting and perhaps season. The robustness of the network can be enhances by different augmentation techniques that is problem dependant. For medical purposes elastic transformations may be beneficial, and in our case lighting
\end{itemize}
\cite{seg}

\subsection{Extension of network}


\subsection{Conclusion}

\bibliographystyle{IEEEbib}
\bibliography{refs}

\subsection{Appendix}
\begin{figure}
	\centering
	\includegraphics[width=0.7\linewidth]{"reconstruction DL"}
	\caption{Reconstruction of the smaller inferences into a unified field prediction.}
	\label{reconstruction_technique}
\end{figure}



\end{document}
